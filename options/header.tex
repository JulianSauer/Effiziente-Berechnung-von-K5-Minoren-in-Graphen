% header.tex

% Korrekte Darstellung der Umlaute (language and fonts)
\usepackage[T1]{fontenc} % T1-Fonts are better
\usepackage{lmodern} % Eine etwas angenehmere Font
\usepackage[utf8]{inputenc} % nutzt UTF8, bitte was sonst
\usepackage[main=ngerman]{babel} % Übersetze einige Literate auf Deutsch

% Math and theoretical computer science symboles/fonts
\usepackage{amsmath}  % American Math Society Packete: general
\usepackage{amsfonts} % American Math Society Packete: more fonts
\usepackage{amssymb}  % American Math Society Packete: more symbols
%\usepackage{amsthm}  % American Math Society Packete: typesetting theorems  %%% in conflict with ntheorem!!!
\usepackage{stmaryrd} % St Mary Road symbols for theoretical computer science
\usepackage{dsfont}   % supports the 1-function

% Caption Packet
\usepackage[margin=0pt,font=small,labelfont=bf]{caption} % cus­tomise the cap­tions in float­ing en­vi­ron­ments
% Gliederung einstellen
%\setcounter{secnumdepth}{5}
%\setcounter{tocdepth}{5}

% citations/ captions/ float­ing en­vi­ron­ments/ links
\usepackage{cite} % Improved Citation-Handling
\usepackage[babel,german=quotes,autostyle]{csquotes} % ad­vanced fa­cil­i­ties for in­line and dis­play quo­ta­tions
\MakeOuterQuote{"}	% for german style quotation marks
\usepackage{subcaption} % Support for sub-captions (and sub-figures)
\usepackage{enumerate} % Erweiterte Enumerate-Umgebungen
\usepackage{wrapfig} % Erlaubt es Text, Graphiken zu umfließen
\usepackage{lscape}
\usepackage{rotating}
%\usepackage{epstopdf}
\usepackage{float} % Verbesserte floating-objects
\usepackage{framed} % Umrahmte Abschnitte
\usepackage[framed,hyperref,amsmath,thmmarks]{ntheorem} % en­hance­ments for the­o­rem-like en­vi­ron­ments
\usepackage[unicode,pdfmenubar,linktoc=all,hidelinks,bookmarks]{hyperref} % ermöglicht PDF-Verklinkungen
\usepackage{url} % al­lows line­breaks at cer­tain char­ac­ters in urls/ e-mail adresses/ paths
\usepackage{multicol} % to use columns with multiple rows

% Abkuerzungen richtig formatieren %
\usepackage{xspace}

% Euro Symbol
\usepackage{eurosym}

% Seitenlayout
\usepackage{geometry} % Beinflusst das Seitenlayout
\geometry{a4paper} % Papiergroesse
%%\geometry{top=25mm, inner=25mm, outer=25mm, bottom=25mm, headsep=10mm, footskip=10mm} % Seitenraender
%%\geometry{left=3.5cm,right=2.5cm,bottom=3.5cm,top=3cm} % Seitenraender
\usepackage{pdflscape} % Landscape Umgebung für PDF Dokumente: set­ting the page at­tribute /Ro­tate 
\usepackage[hang]{footmisc}   % Für den Einzug bei Fußnoten
\setlength{\footnotemargin}{0pt} % Setze den Einzug für die Fußnoten
\parskip 0pt
%\parindent 0pt

% Grafiken
\usepackage[pdftex]{color} % Erweiterter Support für Graphiken
\usepackage{graphicx}
%\usepackage{subfigure}
\usepackage{tikz} % Für LaTeX Graphiken
\usetikzlibrary{automata,arrows,backgrounds,decorations.markings,decorations.pathmorphing,decorations.pathreplacing,fit,positioning,shadows,shapes,shapes.geometric} % Ein paar nützliche TikZ-Pakete


% Bibtex deutsch
\usepackage{bibgerm}

\usepackage{makeidx} % Index für Schlagwörter
\makeindex

\usepackage[final]{listofsymbols}  % Symbolverzeichnis
%\usepackage[draft]{listofsymbols}


% Zeilenabstand einstellen %
\renewcommand{\baselinestretch}{1.25}
% Floating-Umgebungen anpassen %
\renewcommand{\topfraction}{0.9}
\renewcommand{\bottomfraction}{0.8}

% Keine einzelnen Zeilen beim Anfang eines Abschnitts (Schusterjungen)
\clubpenalty = 10000
% Keine einzelnen Zeilen am Ende eines Abschnitts (Hurenkinder)
\widowpenalty = 10000 \displaywidowpenalty = 10000
% EOF