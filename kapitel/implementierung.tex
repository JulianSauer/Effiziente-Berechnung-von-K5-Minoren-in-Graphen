% kapitel5.tex
\chapter{Implementierung}
\label{cha:implementierung}

Im Folgenden werden Details zur Implmentierung im \emph{Open Graph Drawing Framework} erklärt.
Dazu wird zunächst das Framework vorgestellt, anschließend werden wichtige Klassen und Algorithmen behandelt und letztlich zusammengesetzt, um einen zertifierenden Algorithmus zum Finden von \kf-Minoren zu bilden.
\ \\

Das \emph{Open Graph Drawing Framework (OGDF)} ist ein in C++ geschriebenes Framework, das Algorithmen und Datenstrukturen für Graphen enthält, wobei ein besonderes Augenmerk auf dem Zeichnen von Graphen liegt.
OGDF kann unabhängig von anderen Frameworks und Bibliotheken genutzt werden und läuft sowohl unter Linux als auch unter Windows und MacOS.
Es wurde 2005 unter der GNU General Public License als Open Source Projekt veröffentlicht\cite{OGDFAbout}\cite{CGJK+2014}.

\newpage
\begin{minipage}{\linewidth}
\lstinputlisting[language=C++,
                 frame=single,
                 keywordstyle=\color{blue},
                 basicstyle=\footnotesize,
                 numbers=left,
                 caption={Implementierung des Algorithmus von Kezdy und McGuinness.},
                 captionpos=b]
                {algorithmen/FindK5Minor.cpp}
\end{minipage}
