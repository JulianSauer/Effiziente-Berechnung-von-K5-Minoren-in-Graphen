% kapitel5.tex
\chapter{Implementierung}
\label{cha:implementierung}

Im Folgenden werden Details zur Implmentierung im \emph{Open Graph Drawing Framework} erklärt.
Dazu wird zunächst das Framework vorgestellt, anschließend wird erklärt, wie der Algorithmus von Kezdy und McGuinness zusammen mit einer Wagner-Struktur umgesetzt werden kann, um ein Zertifikat zu liefern.
\ \\

Das \emph{Open Graph Drawing Framework (OGDF)} ist ein in C++ geschriebenes Framework, das Algorithmen und Datenstrukturen für Graphen enthält, wobei ein besonderes Augenmerk auf dem Zeichnen von Graphen liegt.
OGDF kann unabhängig von anderen Frameworks und Bibliotheken genutzt werden und läuft sowohl unter Linux als auch unter Windows und MacOS.
Es wurde 2005 unter der GNU General Public License als Open Source Projekt veröffentlicht\cite{OGDFAbout}\cite{CGJK+2014}.

\section{Algorithmus von Kezdy und McGuinness als Wagner-Struktur}
Im Folgenden wird die Wagner-Struktur mit dem Algorithmus von Kezdy und McGuinness verglichen und dieser so modifiziert, dass er sie erzeugen kann.
Allerdings werden augmentierte Komponenten für $1$- und $2$-Separatoren durch Block-Cut Trees und SPQR-Bäume erzeugt sowie ein zertifizierender Algorithmus beschrieben.

\subsection{Block-Trees}
Im Algorithmus von Kezdy und McGuinness wird der Graph in augmentierte Komponenten aufgeteilt, wenn ein $(1, 2)$-, $(2, 2)$- oder \dd-Separator gefunden wird.
Er wird so lange ausgeführt, bis die Blockknoten aller \dd-Block-Trees der Wagner-Struktur planar \bzw isomorph zu $W$ sind oder ein \kf-Minor gefunden wurde.
Um eine Wagner-Struktur aufzubauen, wird zunächst für jede Zusammenhangskomponente $G$ des Eingabegraphen ein $1$-Block-Tree angelegt, der einen einzelnen auf $G$ verweisenden Blockknoten $v_B$ enthält.
Eine Funktion, um einen $1$-Separator $C$ in $G$ zu finden, existiert bereits in OGDF.
Sind alle Graphen der Blockknoten $2$-zusammenhängend, kann mit dem nächsten Schritt fortgefahren werden.
Andernfalls werden die augmentierten Komponenten $A_1, ..., A_j$ gebildet, die durch den gefundenen $1$-Separator $C$ in $G$, auf den der Blockknoten $v_B$ verweist, definiert werden.
Dann wird $v_B$ durch einen Cliquenknoten $v_C$ für $C$ sowie einem Blockknoten $v_{b_{A_i}}$ für jede augmentierte Komponente ersetzt.
Anschließend werden Kanten der Form $(v_C, v_{b_{A_i}})$ hinzugefügt, um jede neue augmentierte Komponente mit dem neuen Cliquenknoten zu verbinden.
War $v_B$ adjazent zu einem anderen Cliquenknoten $v_C'$, der auf einen Teilgraph $G_C$ verweist, dann gibt es jetzt genau eine augmentierte $v_{b_{A_i}}$, die den Teilgraph $G_C$ enthält.
Die Kante $(v_C', v_{b_{A_i}})$ wird hinzugefügt.
Letztlich kann dieser Schritt auf alle neu entstandenen augmentierten Komponenten rekursiv angewendet werden, bis keine $1$-Separatoren mehr gefunden werden.

Das gleiche Verfahren kann genutzt werden, um $2$-Block-Trees für alle Blockknoten des $1$-Block-Trees zu erzeugen -- eine Suche nach $2$-Separatoren ist ebenfalls bereits implementiert.
Wird ein Cliquenknoten im $2$-Block-Tree angelegt, muss \ggf eine zusätzliche Kante im assoziierten Graph eingefügt werden, damit die beiden Separatorknoten eine Clique formen.
Da die Blockknoten die augmentierten Komponenten enthalten, ist die Kante dort immer vorhanden.

Als nächstes wird ein \dd-Block-Tree $T$ für jeden Blockknoten $v_B$ der $2$-Block-Trees aufgebaut.
Zunächst wird ein neuer Blockknoten in $T$ angelegt, der auf den mit $v_B$ assozierten Graph $G$ verweist.
Der bereits implementierte Planaritätstest von Boyer und Myrvold \cite{BoM04} kann auf $G$ angewendet werden.
Wird dadurch ein \kf-Minor gefunden, kann der Algorithmus stoppen -- die Wagner-Struktur ist ungültig und der Eingabegraph enthält einen \kf-Minor.
Wird in $G$ ein \kdd-Minor gefunden, wird geprüft, ob $G$ isomorph zu $W$ ist.
Ist das der Fall, dann enthält $T$ zu diesem Zeitpunkt immer genau einen Blockknoten, der auf den zu $W$ isomorphen Graph verweist -- hätte $T$ weitere Knoten, dann hätte nach Lemma \ref{eq:Lemma36} bereits ein Rekursionsschritt zuvor ein \kf-Minor gefunden werden müssen.
Somit ist $T$ bei Isomorphie zu $W$ immer ein gültiger \dd-Block-Tree.
Andernfalls wird geprüft ob drei Knoten des \kdd-Minors einen \dd-Separator $C$ formen.
Falls ja, wird analog zu den anderen beiden Block-Trees $G$ in augmentierte Komponenten aufgeteilt und der \dd-Block-Tree entsprechend angepasst.
Erneut ist darauf zu achten, dass die Knoten von $C$ als Clique in dem Graph gespeichert werden, mit dem der neue Cliquenknoten assoziiert wird.
Anschließend wird der Planaritätstest rekursiv auf die augmentierten Komponenten angewendet.
Bildet der \kdd-Minor dagegen keinen \dd-Separator, dann kann nach den Lemmata aus Kapitel \ref{cha:algorithmuskezdymcguinness} ein \kf-Minor konstruiert werden und der Algorithmus stoppen.
Findet der Planaritätstest in keinem Graphen der Blockknoten von $T$ weitere \kf- oder \kdd-Minoren, dann ist $T$ ein gültiger \dd-Block-Tree, der aus planaren Blöcken besteht.
Wurden alle \dd-Block-Trees aufgebaut, sodass die Wagner-Struktur gültig ist, dann ist der Eingabegraph \kf-Minor-frei.

\subsection{Block-Cut Trees und SPQR-Bäume}
Wie in Kapitel \ref{cha:wagnerstruktur} angedeutet, werden statt $1$-Block-Trees Block-Cut Trees und statt $2$-Block-Trees SPQR-Bäume verwendet werden.
Für beide gibt es bereits Implementierungen in OGDF, die in Linearzeit laufen.
Der Nachteil ist, dass wenn ein $1$-Block-Tree durch einen Block-Cut Tree aufgebaut wird, jedes adjazente Knotenpaar des Block-Cut Trees nach dem gemeinsamen Knoten durchsucht werden muss, der den $1$-Separator bildet.
Vor allem für einen $2$-Block-Tree gestaltet sich die Suche nach $2$-Separatoren als schwierig, da die Graphen der Blockknoten des $2$-Block-Trees nicht identisch zu den Komponenten der R-Knoten des SPQR-Baumes sind.
Ohne weitere Anpassungen scheint es nicht möglich, die $2$-Separatoren aus dem SPQR-Baum direkt zu gewinnen.
Stattdessen kann auf die $3$-Zusammenhangssuche in OGDF zurückgegriffen werden, um $2$-Separatoren zu finden.
Sie brachte in der Implementierung jedoch weitere Schwierigkeiten mit sich, da sie nicht auf unzusammenhängenden Graphen funktioniert.
Außerdem wird pro Suchdurchlauf nur ein $2$-Separator gefunden, sodass durch mehrfaches Ausführen bereits in diesem Schritt eine quadratische Laufzeit entstehen kann.
Wie im nächsten Kapitel beschrieben, sind die Cliquenknoten relevant für das Zertifikat.
Die experimentelle Analyse in Kapitel \ref{cha:analyse} testet dagegen vor allem die Laufzeit des Algorithmus -- die teils aufwändige Zertifikatsberechnung wurde nur für \kf-Minor-freie Graphen umgesetzt.


\section{Zertifierender Algorithmus}
Nachdem im vorherigen Abschnitt erklärt wurde, wie der Algorithmus entscheidet, ob ein \kf-Minor enthalten ist, wird nun darauf eingegangen, wie die vom Algorithmus getroffene Aussage zertifiziert werden kann.

\subsection{Virtuelle Kanten}
Werden für einen $2$- oder \dd-Separator $C$ in $G$ augmentierte Komponenten erzeugt, enthält jede Komponente die Knoten von $C$ als Clique.
Für den Fall, dass die Knoten in $G$ keine Clique formen, wurden also Kanten hinzugefügt, die als \emph{virtuell} bezeichnet werden.
Neben der Beantwortung der Frage, ob ein \kf-Minor enthalten ist, ist es für einige Anwendungsfälle wie das Ausgeben eines Zertifikates interessant, diesen in $G$ zu finden:
Das \emph{Modell} eines \kf-Minoren in $G$ besteht aus fünf Teilgraphen, die in $G$ zusammenhängend und paarweise durch mindestens einen Pfad verbunden sind.
Dafür ist es nicht nur wichtig, virtuelle Kanten zu markieren, sondern auch zu wissen, aus welchen Pfaden sie in $G$ bestehen.
Wie im Beweis von Theorem \ref{eq:Theorem33} bereits festgestellt, korrespondiert jede virtuelle Kante zu einem Pfad:
Sei beispielsweise $\{u, v\}$ ein $2$-Separator, der einen zusammenhängenden Graph $G$ in zwei Zusammenhangskomponenten $Z_1$ und $Z_2$ teilt.
Ist $A_1$ die augmentierte Komponente, die den Teilgraph $Z_1 \cup \{u, v\}$ enthält, dann ist die Kante $(u, v)$ ein Pfad in $G$, der durch den Teilgraph $Z_2$ verläuft.

Um etwa den Teilgraph in $G$ zu finden, für den in einem Blockknoten eines \dd-Block-Trees ein \kf-Modell $K$ gefunden wurde, müssen die Knoten und Kanten von $K$ auf $G$ abgebildet werden:
\begin{definition}
  Sei $G = (V_G, E_T)$ ein Graph und $G_T = (V_T, E_T)$ der Graph, der zu einem Block- oder Cliquenknoten eines Block-Trees von $G$ gehört.
  Dann ist jeder Knoten $v_T \in V_T$ aus genau einem Knoten $v_G \in V_G$ durch (wiederholtes) Kopieren entstanden.
  Dies sei dargestellt durch $v_G = V_G[v_T]$ \bzw für Mengen $\{v_{T_1}, v_{T_2}\} \subseteq V_T$ durch $\{v_{G_1}, v_{G_2}\} = V_G[\{v_{T_1}, v_{T_2}\}]$.
  Für alle $v_{T_1}, v_{T_2} \in V_T$ gilt $V_G[v_{T_1}] \neq V_G[v_{T_2}]$.
\end{definition}
Jeder Knoten von $K$ kann also auf genau einen Knoten in $G$ abgebildet werden.
Um den Pfad zu finden, aus dem eine virtuelle Kante in $K$ besteht, kann die Wagner-Struktur benutzt werden.
Zwar ist sie ungültig in dem Sinne, dass der Graph nicht \kf-Minor-frei ist.
Allerdings bilden die $2$- und \dd-Block-Trees durch die zu Cliquenknoten adjazenten Blockknoten ab, welche Pfade in $G$ benutzt werden können.
Sei $T = (V_T, E_T, \mathcal{G})$ ein \dd-Block-Tree, der einen Blockknoten $t \in V_T$ mit Graph $G_t$ enthält.
Hat $G_t$ eine virtuelle Kante $e_v = (c_1, c_2)$, dann gibt es entweder einen Cliquenknoten $t_c$ mit Graph $G_{t_c}$, sodass $(t, t_c) \in E_T$ und $G_{t_c}$ die Knoten $\{c_1, c_2\}$ hat.
Oder es gibt einen Cliquenknoten in dem $2$-Block-Tree, der $T$ als Blockknoten enthält.
Wird zu jeder virtuellen Kante eine Referenz zu dem Cliquenknoten gespeichert, für den sie erzeugt wurde, ist das Verfahren analog und wird für einen Cliquenknoten des \dd-Block-Trees erklärt:
Es gibt einen Blockknoten $t' \in V_T$ mit Graph $G_t'$, der adjazent zu $t_c$ ist und $t \neq t'$.
Um den Pfad in $G$ zu finden, den $e_v$ darstellt, kann ein beliebiger Pfad $P(c_1, c_2) = \{(c_1, v_1), (v_2, v_3), ..., (v_k, c_2)\}$ in $G_t'$ gesucht werden, sodass $(c_1, v_1)$ und $(v_k, c_2)$ keine virtuellen Kanten sind.
Dass es einen solchen Pfad gibt, wurde in Theorem \ref{eq:Theorem33} gezeigt.
Ist $e_v' = (v_i, v_j) \in P(c_1, c_2)$ eine virtuelle Kante, sodass sie weder inzident zu $c_1$ noch zu $c_2$ ist, dann gibt es einen Blockknoten, dessen Graph den Pfad $P(v_i, v_j)$ hat.
Durch Rekursion können die Schritte solange angewendet werden, bis $P(c_1, c_2)$ keine virtuellen Kanten enthält.


\subsection{Zertifikat}
Ist $G$ der Eingabegraph, dann besteht das Zertifikat grundsätzlich aus Teilgraphen von $G$.
Einfachheitshalber wird angenommen, dass keine virtuellen Kanten enthalten sind.
Ist der Graph $G$ \kf-Minor-frei, dann kann die aufgebaute Wagner-Struktur als Zertifikat dienen, um die Aussage nach dem Theorem von Wagner \cite{Wag37} zu belegen.
Sei $G_T = (V_T, E_T)$ der Graph, der mit dem Block- oder Cliquenknoten eines Block-Cut Trees assoziiert wird und $G = V_G, E_T$ der Graph, der als Eingabe benutzt wurde.
Es werden die Blockknoten aller \dd-Block-Trees sowie die Cliquenknoten aller Block-Trees der Wagner-Struktur wie folgt geprüft:
\begin{enumerate}
  \item \textbf{Planarer Blockknoten}: Sei $G'$ für $G$ folgender Teilgraph: $G' = G(V_G[V_T])$
        Dann ist $G'$ planar, was \zB durch einen Planaritätstest geprüft werden kann.
  \item \textbf{Zu $W$ isomorpher Blockknoten}: Es gilt $\vert V_T \vert = 8$.
        Der Teilgraph $G' = G(V_G[V_T])$ besteht also aus acht Knoten, die in $G$ eine $W$-Subdivision formen.
        Der für den Algorithmus von Kezdy und McGuinness implementierte Isomorphietest kann in Kombination mit \zB einer Tiefensuche zum Prüfen verwendet werden.
  \item \textbf{Cliquenknoten}: Die Knotenmenge $V_G[V_T]$ bildet in $G$ einen Separator.
        Sei $a$ die Anzahl der Zusammenhangskomponenten von $G$.
        Für $\vert V_T \vert \leq 2$ enthält $G - V_G[V_T]$ mindestens $a+1$ Zusammenhangskomponenten.
        Für $\vert V_T \vert = 3$ enthält $G - V_G[V_T]$ mindestens $a+2$ Zusammenhangskomponenten.
\end{enumerate}

Andernfalls kann ein \emph{Modell} des gefunden \kf-Minoren zur Verifikation dienen.
Wird jeder der fünf Teilgraphen zu einem Knoten kontrahiert und jeder Pfad, der zwei dieser Knoten verbindet, zu einer Kante kontrahiert, dann entsteht nach dem Entfernen von Mehrfachkanten ein Graph, der isomorph zu \kf ist.
So kann \zB durch eine Tiefensuche leicht geprüft werden, ob jeder der fünf Teilgraphen in $G$ zusammenhängend ist, und ob sie paarweise miteinander verbunden sind.

% TODO
\newpage
\begin{minipage}{\linewidth}
\lstinputlisting[language=C++,
                 frame=single,
                 keywordstyle=\color{blue},
                 basicstyle=\footnotesize,
                 numbers=left,
                 caption={Implementierung des Algorithmus von Kezdy und McGuinness.},
                 captionpos=b]
                {algorithmen/FindK5Minor.cpp}
\end{minipage}
