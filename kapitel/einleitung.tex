% einleitung.tex
\chapter{Einleitung}
\label{cha:einleitung}

Im Rahmen dieser Masterarbeit wird ein Algorithmus erklärt und implementiert, der entscheiden kann, ob ein Graph \kf-Minor-frei ist.
Er basiert auf einem von Kezdy und McGuinness \cite{KeM92} vorgestellten Algorithmus, der quadratische Laufzeitkomplexität besitzt.

\section{Motivation und Hintergrund}
\label{sec:motivation_und_hintergrund}
Durch die Berechnung von \kf-Minoren kann entschieden werden, ob ein Graph \kf-Minor-frei ist.
Das ist insofern interessant, als dass einige Algorithmen effizienter auf \kf-Minor-freien Graphen ausgeführt werden können.
So ist die Berechnung eines \emph{maximalen Schnittes} (\emph{Max-Cut Problem}) - also das Aufteilen der Knoten eines Graphen in zwei Mengen, sodass die Kanten zwischen diesen beiden Mengen in Summe ein maximales Gewicht besitzen - NP-schwer\cite{Kar72}.
Allerdings wird etwa in \cite{Bar83} gezeigt, dass es für Graphen ohne \kf-Minor in Polynomialzeit gelöst werden kann.
Viele kombinatorische Optimierungsprobleme wie quadratische 0-1 Probleme können als Max-Cut Probleme formuliert werden\cite{BJR89}, sodass \kf-Minor-freie Probleminstanzen deutlich effizienter gelöst werden können.

In \cite{JLMR+19} transformieren Jünger et al.\nolinebreak[4]\@\xspace Ising Spin Glass Probleme zu Max-Cut Problemen, um mit Hilfe von einem Branch-and-Cut Algorithmus eine optimale Lösung zu finden.
Diese kann mit den heuristischen Lösungen, die etwa auf dem Quantencomputer \emph{D-Wave 2000Q} berechnet werden, verglichen werden.
Dadurch kann beispielsweise bewertet werden, ob für die schnellere Laufzeit des Quantencomputers die heuristische Lösung tragbar ist.
Innerhalb des exakten Verfahrens stellen die \kf-Minoren die Schwierigkeit des Problems dar.
Deshalb kann das Problem einerseits für \kf-Minor-freie Graphen effizient gelöst werden.
Andererseits können einer oder mehrere gefundene \kf-Minoren benutzt werden, um innerhalb des linearen Programms den Suchraum zu beschränken und die Berechnung zu beschleunigen.

\section{Aufbau der Arbeit}
\label{sec:aufbau}
Zunächst werden einige Definition gegeben, die anschließend benutzt werden, um den Algorithmus von Kezdy und McGuinness \cite{KeM92} vorzustellen.
Anschließend wird ein auf Wagner \cite{Wag37} zurückzuführendes Strukturtheorem der beiden Autoren genauer betrachtet.
Für den Fall, dass ein Graph \kf-Minor-frei ist, kann dadurch ein Zertifikat erzeugt werden, über dass die getroffene Aussage verifiziert werden kann.
Letztlich beschäftigt sich die Arbeit mit der Implementierung des Algorithmus im \emph{Open Graph Drawing Framework} sowie einer experimentellen Analyse dieser Implementierung.
