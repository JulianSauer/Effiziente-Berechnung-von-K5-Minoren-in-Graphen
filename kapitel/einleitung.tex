% einleitung.tex
\chapter{Einleitung}
\label{cha:einleitung}

\section{Motivation und Hintergrund}
\label{sec:motivation_und_hintergrund}
Zunächst kann durch die Berechnung von \kf-Minoren entschieden werden, ob ein Graph \kf-Minor frei ist.
Das ist insofern interessant, als das einige Algorithmen effizienter auf solchen Graphen ausgeführt werden können.
So ist die Berechnung eines \emph{maximalen Schnittes} (\emph{Max-Cut Problem}) - also das Aufteilen der Knoten eines Graphen in zwei Mengen, sodass die Kanten zwischen diesen beiden Mengen in Summe ein maximales Gewicht besitzen - NP-schwer\cite{Kar72}.
Allerdings wird etwa in \cite{Bar83} gezeigt, dass es für Graphen ohne \kf-Minor in Polynomialzeit gelöst werden kann.
Darüber hinaus können viele kombinatorische Optimierungsprobleme wie quadratische 0-1 Probleme als Max-Cut Probleme formuliert werden\cite{BJR89}.
In \cite{JLMR+19} transformieren Jünger et al.\nolinebreak[4]\@\xspace Ising Spin Glass Probleme zu Max-Cut Problemen, um mit Hilfe von einem Branch-and-Cut Algorithmus eine optimale Lösung zu finden.
Neben diesem Ansatz können Ising Spin Glass Probleme ebenfalls auf Quantencomputern wie dem D-Wave 2000Q heuristisch, aber dafür schneller gelöst werden.
Zur Bewertung ihrer Qualität können sie mit denen von Jünger et al.\nolinebreak[4]\@\xspace verglichen werden.
Innerhalb von diesem exakten Verfahren stellen die \kf-Minoren die Schwierigkeit des Problems dar.
Von daher ist es einerseits interessant für die Berechnung, wenn kein \kf-Minor enthalten ist, sodass das Problem effizient gelöst werden kann.
Andererseits können einer oder mehrere gefundene \kf-Minoren benutzt werden, um innerhalb des linearen Programms den Suchraum zu beschränken und die Berechnung zu beschleunigen.

\section{Aufbau der Arbeit}
\label{sec:aufbau}
